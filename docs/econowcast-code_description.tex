
\documentclass[12pt]{article}
%%%%%%%%%%%%%%%%%%%%%%%%%%%%%%%%%%%%%%%%%%%%%%%%%%%%%%%%%%%%%%%%%%%%%%%%%%%%%%%%%%%%%%%%%%%%%%%%%%%%%%%%%%%%%%%%%%%%%%%%%%%%%%%%%%%%%%%%%%%%%%%%%%%%%%%%%%%%%%%%%%%%%%%%%%%%%%%%%%%%%%%%%%%%%%%%%%%%%%%%%%%%%%%%%%%%%%%%%%%%%%%%%%%%%%%%%%%%%%%%%%%%%%%%%%%%
\usepackage{float}
\usepackage[dvipsnames]{xcolor}
\usepackage{engord}
\usepackage{subfig}
\usepackage{amssymb}
\usepackage{amsmath}
\usepackage{natbib}
\usepackage{portland}
\usepackage[format=default,singlelinecheck=true,justification=centering]{caption}
\usepackage[onehalfspacing]{setspace}
\usepackage{geometry}
\usepackage{scalefnt}
\usepackage{theorem}
\usepackage{lscape}
\usepackage{graphicx}
\usepackage{rotating}
\usepackage{amsfonts}
\usepackage{fancyhdr}
\usepackage{array}
\usepackage{url}
\usepackage{listings}
\usepackage[T1]{fontenc}

\setcounter{MaxMatrixCols}{10}
%TCIDATA{OutputFilter=Latex.dll}
%TCIDATA{Version=5.50.0.2960}
%TCIDATA{<META NAME="SaveForMode" CONTENT="1">}
%TCIDATA{BibliographyScheme=Manual}
%TCIDATA{LastRevised=Sunday, November 12, 2017 10:08:45}
%TCIDATA{<META NAME="GraphicsSave" CONTENT="32">}

\newtheorem{code}{Code}
\newcommand\BeraMonottfamily{  \def\fvm@Scale{0.85}  \fontfamily{fvm}\selectfont}
\renewcommand{\thefootnote}{\arabic{footnote}}
\newtheorem{assumption}{Assumption}
\addtocounter{page}{-1}
\pagestyle{fancy}
\fancyhf{}
\rhead{European Commission Eurostat}
\lhead{\includegraphics[height=1cm, keepaspectratio=true]{E://Dropbox/BD2/GOPA-logo.png}}
\rfoot{\thepage}
\newcolumntype{L}[1]{>{\raggedright\let\newline\\\arraybackslash\hspace{0pt}}m{#1}}
\newcolumntype{C}[1]{>{\centering\let\newline\\\arraybackslash\hspace{0pt}}m{#1}}
\newcolumntype{R}[1]{>{\raggedleft\let\newline\\\arraybackslash\hspace{0pt}}m{#1}}
\lstset{
    language=R,
    keepspaces=true,
    basicstyle=\fontsize{8}{8}\ttfamily,
    breaklines=true,
    frame=tb,     tabsize=4,     showstringspaces=false,     numbers=left,     numberstyle=\tiny\color{gray},
    commentstyle=\color{ForestGreen},     keywordstyle=\color{blue},     stringstyle=\color{red} }
\renewcommand{\lstlistingname}{Code}
\renewcommand{\lstlistlistingname}{List of \lstlistingname s}

\input{tcilatex}

\begin{document}

\title{\textbf{Big Data Econometrics}\\
\textbf{Nowcasting and Early Estimates}\\
\textbf{Task 10: ``Big data handling tool developed as R package including
supporting user documentation for deployment"}}
\author{George Kapetanios\thanks{%
kapetaniosgeorge@gmail.com} \and Massimiliano Marcellino\thanks{%
massimiliano.marcellino@unibocconi.it} \and Fotis Papailias\thanks{%
fotis.papailias@kcl.ac.uk; fotis.papailias@quantf.com} \and Katerina Petrova%
\thanks{%
katerina.petrova@st-andrews.ac.uk} }
\maketitle

\begin{abstract}
Functions and usability of the R code are discussed in the paper. Most of the
code is originally written by the authors. In cases an R package is used, it
is cited appropriately.

\textit{Keywords: Big Data, Nowcasting, Density Forecasting, Density
Evaluation, R Software.}
\end{abstract}

\newpage
\clearpage
\tableofcontents
\newpage

\section{Task 2}

\subsection{High Frequency Returns. File name: \color{blue}{HF.R}}

Here we use the \emph{``highfrequency"} R package, load edit and plot the
necessary data as follows. We use various comments in order to explain the
code and the inputs used in each function.

\begin{lstlisting}[title=\textbf{Realised Volatility based on 5-min cleaned returns.}]
library("highfrequency")

# Load data
data(sample_returns_5min)

# Store the 5 min returns data
r <- sample_returns_5min

# Calculate the realised volatility based on 5 min returns
# Input. r: 5mins return data
RV <- rowSums(r^2)

# Creat a plot
# Input. RV: realised volatility based on 5mins return data as calculated above
plot(RV, type="l", main="EUR/USD 5-min RV", xlab="Time", ylab="RV")
\end{lstlisting}

\subsection{Mobile Phone Data. File name: \color{blue}{MobPhone.R}}

Here we load some .csv files which are also provided. The purpose of this
code is to load the mobile phone activity data and aggregate it in terms of
calls, SMS and internet activity. We use various comments in order to
explain the code and the inputs used in each function.

Below we only demonstrate the part of the code which corresponds to Internet
activity. The same procedure is replicated for call and SMS activity.

\begin{lstlisting}[title=\textbf{Mobile Phone Data Aggregation and Plots.}]
# Load the necessary data.
# Each file corresponds to a single day.
days <- c("sms-call-internet-mi-2013-11-01.csv",
          "sms-call-internet-mi-2013-11-02.csv",
          "sms-call-internet-mi-2013-11-03.csv",
          "sms-call-internet-mi-2013-11-04.csv",
          "sms-call-internet-mi-2013-11-05.csv",
          "sms-call-internet-mi-2013-11-06.csv",
          "sms-call-internet-mi-2013-11-07.csv")

# Create some labels to be used in the plots later		
days.l <- c("2013-11-01, Friday", "2013-11-02, Saturday", "2013-11-03, Sunday",
            "2013-11-04, Monday", "2013-11-04, Tuesday", "2013-11-05, Wednesday",
            "2013-11-07, Thursday")
days.l2 <- c("2013-11-01", "2013-11-02", "2013-11-03",
            "2013-11-04", "2013-11-04", "2013-11-05",
            "2013-11-07")

# Create a sparse matrix to store the results	
result1 <- array(NA, dim=c(24,5,NROW(days)))

# We start with a loop.
# j runs for each different day. In the above we have 7 days (NROW(days)).
for(j in 1:NROW(days))
{
  # Load the data for day j.
  x <- read.csv(days[j], header=TRUE)

  # Identify unique users
  ud <- unique(x[,1])

  # Extract Total activity during the day across users
  # Create a sparse matrix to store the results
  totact <- matrix(NA, NROW(ud), 5)
  rownames(totact) <- as.character(ud)
  colnames(totact) <- c("smsin","smsout","callin","callout","internet")

  for(i in 1:NROW(ud))
  {
    it <- ud[i]
    choose <- which(x[,1]==it)
	# Input. x which has the data and here we are looping across users.
    totact[i,1] <- sum(x[it, 4], na.rm=TRUE)
    totact[i,2] <- sum(x[it, 5], na.rm=TRUE)
    totact[i,3] <- sum(x[it, 6], na.rm=TRUE)
    totact[i,4] <- sum(x[it, 7], na.rm=TRUE)
    totact[i,5] <- sum(x[it, 8], na.rm=TRUE)
  }
  result1[,,j] <- totact
  cat("day ", j, " just done - still left ", NROW(days), "\n")
}

# We are now ready to create some plots
# First, generate the colours.
cols <- rainbow(NROW(days), alpha = 1)

# Calls Internet
# Using the 5th column of array across days, "i", we can aggregate (sum) the internet activity.
i <- 1
plot(result1[,5,i], type="l", xlab="Hours", ylab="Call", col=cols[i],
     main="Total Internet Activity")
# Add the remaining days looping across "i"
for(i in 2:NROW(days))
{
  lines(result1[,5,i], col=cols[i])
}
legend("topleft", legend=days.l, col=cols, lty=rep(1, NROW(days)),
       cex=0.6)
\end{lstlisting}

\subsection{Web Prices. File name: \color{blue}{Prices.R}}

Here we load a .csv sample file with the appropriate data. We use various
comments in order to explain the code and the inputs used in each function.

\begin{lstlisting}[title=\textbf{Web Prices Aggregation and Plots.}]
# Load the data
x <- read.csv("arg-sample2.csv", header=TRUE)
x <- as.matrix(x)
x <- x[,c(5, 1, 7, 6)] # Extract the necessary columns
colnames(x) <- c("date","id", "cat", "price")

d <- as.Date(x[,1])		# create the dates sequence
ud <- sort(unique(d))	# extract unique dates for time series aggregation
uid <- unique(x[,2])	# extract unique id's
uc <- unique(x[,3])		# extract unique categories

# Create sparse matrix to store the results
cats <- matrix(NA, NROW(ud)-1, NROW(uc))
colnames(cats) <- uc
rownames(cats) <- as.character(ud[2:NROW(ud)])

# Start the loop across categories
for(i in 1:NROW(uc))
{
  xcat <- which(x[,3]==uc[i])
  xcat <- x[xcat,]

  tmat <- matrix(NA, NROW(ud), NROW(uid))
  colnames(tmat) <- c(uid)

  # Create a doouble loop: (j) across id's and (jj) across dates
  for(j in 1:NROW(uid))
  {
    xcat2 <- which(xcat[,2]==uid[j])
    xcat2 <- xcat[xcat2,]

    for(jj in 1:NROW(ud))
    {
      cw <- which(xcat2[,1]==ud[jj])

      if(NROW(cw)==0){
        tmat[jj,j] <- NA
      }else{
        tmat[jj,j] <- xcat2[cw,4]
      }
    }
  }
  # sure that the result is numeric
  tmat <- apply(tmat, 2, as.numeric)

  # Apply the methodology of Cavallo
  R <- tmat[2:NROW(tmat),]/tmat[1:(NROW(tmat)-1),]
  R <- apply(R, 1, prod, na.rm=TRUE)
  R <- R^(1/NCOL(tmat))

  cats[,i] <- R
}

# Cumulate across time
I <- apply(cats, 2, cumprod)
w <- as.matrix(rep(1/NCOL(I), NCOL(I)))

# use the approprite weights as in Cavallo
S <- I %*% w

# Produce the final CPI estimate plot.
plot(as.Date(rownames(S)), S, type="l", main="Online Prices CPI", xlab="Time", ylab="Index")
\end{lstlisting}

\subsection{Google Trends. File name: \color{blue}{Google.R}}

We use the \emph{``gtrendsR"} R library to download Google Trends in R.
Currently, Google Trends are offered in monthly frequency.

\begin{lstlisting}[title=\textbf{Google Trends Download}]
# User Input
kwd <- "gbp"         	# Keyword(s)
reg <- "GB"             # Region
tsd <- "all"            # Time Frame: "all" (since 2004),
                        # "today+XXX-y" (last XXX years)
stp <- "web"            # "web", "news", "images", "froogle", "youtube"
cct <- 0                # category

# Call the function and download data
gt <- gtrends(kwd, reg, tsd, stp, cct)

# Make a plot
plot(gt)
\end{lstlisting}

\subsection{Twitter Data. File name: \color{blue}{Twitter.R}}

We use the \emph{``twitteR"} R library to download Twitter data in R.

\begin{lstlisting}[title=\textbf{Twitter Data}]
# Input: user defined keyword, below we use #gbpusd feed
#        n: the number of nodes to download
rdmTweets <- searchTwitter('#gbpusd', n=6500)
\end{lstlisting}

\subsection{Reuters written in Python}

Lines:

\begin{itemize}
\item 1-5: import packages: Beautifulsoup (HTML parsing), Scrapy (web
crawling), Logging (recording errors), DateTime (parsing dates);

\item 7: open a log file to record 404 errors (page not found);

\item 11: create a name for the web scraper within the Scrapy CrawlSpider
class;

\item 14-19: spider settings. Note that we disabled the ``AUTOTHROTTLE''
setting and defined parameters manually to limit speed (one page download
every 0.2 seconds) and contemporaneous requests (4).

\item 21-26: define the input URLs for the web scraper. In a typical Scrapy
crawler, one would only define a single starting page for the spider to move
from, but we have already obtained the full list of pages to scrape in a
text file. Therefore, we copy all the URLs in the file to a list and we pass
each item to the ``parse'' function below.

\item 29-31: writes to a log file 404 errors;

\item 33: passes the HTML code of the page to the parsing library \texttt{%
lxml};

\item 34-35: finds and isolates the permanent URL of the article;

\item 36-37: finds and isolates the alternate URL of the article;

\item 38-39: finds and isolates the article tags;

\item 40-42: finds and isolates the publication date, headline and text of
the article;

\item 43-51: removes the remaining HTML formatting for clean reading;

\item 52-53: parses the date and publication time to obtain a short-form
date;

\item 55-57: writes the collected data to a CSV file.
\end{itemize}

\begin{lstlisting}[language=Python]

from bs4 import BeautifulSoup
import scrapy
import logging
from scrapy.spiders import CrawlSpider
from datetime import datetime

logging.basicConfig(filename='log_50.log', level=logging.ERROR)


class SpiderReuters (CrawlSpider):
    name = 'crawler_final_v21_tags_2'
    handle_httpstatus_list = [404]

    custom_settings = {
        'AUTOTHROTTLE_ENABLED': False,
        'CONCURRENT_REQUESTS_PER_DOMAIN': '4',
        'DOWNLOAD_DELAY': '0.2',
        'USER_AGENT': "Mozilla/5.0 (X11; Ubuntu; Linux x86_64; rv:38.0) Gecko/20100101 Firefox/38.0"
    }

    def start_requests(self):
        f = open(r"/home/PATH-to-file/list_of_URLs.txt", 'r')
        start_urls = [url.strip() for url in f.readlines()]
        f.close()
        for urls in start_urls:
            yield scrapy.Request(url=urls, callback=self.parse)

    def parse(self, response):
        if response.status == 404:
            with open('log_404.txt', 'a') as l:
                l.write(str(response) + '\n')
        else:
            soup = BeautifulSoup(response.body, 'lxml')
            extractor_a = soup.find('link', rel='canonical')
            a_return = extractor_a['href']
            extractor_alt = soup.find('link', rel='alternate')
            alt_return = extractor_alt['href']
            extractor_tags = soup.find('meta', property="og:article:tag")
            tags_return = extractor_tags['content']
            extractor_1 = soup.find_all('span', class_='timestamp')
            extractor_2 = soup.find_all('h1', class_='article-headline')
            extractor = soup.find_all('span', id='article-text')
            composer = [extractor_1, a_return, alt_return, extractor_2, extractor, tags_return]
            composer_2 = []
            for element in composer:
                element_2 = ' '.join(str(element).strip('[]').splitlines())
                composer_2.append(element_2)
            composer_text = "|".join([str(item).replace("|", "") for item in composer_2])
            clean_page = BeautifulSoup(composer_text, 'lxml')
            all_text = ' '.join(clean_page.findAll(text=True))
            all_text_uni = all_text.decode('unicode_escape').encode('utf-8', 'strict')
            date = all_text_uni[4:17].replace(',', '').strip(' ')
            date_object = datetime.strptime(date, '%b %d %Y')
            page = str(response).replace("/", "").replace(".", "").replace(":", "").replace("<", "").replace(">", "")
            with open(r'/home/PATH/output_%s.csv' % page, 'a') as a:
                a.write(date_object.strftime('%b %d %Y') + '|')
                a.write(' '.join(all_text_uni.splitlines()).replace("  ", " "))

\end{lstlisting}

\subsubsection{Index construction}

\begin{itemize}
\item Lines 1-5: imports packages: time (algorithm timer), os (interfacing
with operating system), re (regular expressions), csv (comma separatted
values file reading and writing), pandas (time series statistical package);

\item 7: starts timer;

\item 9: initializes article counter;

\item 11: dictionary of search terms (regular expressions);

\item 13-17: creates a CSV file for articles with column headers 'date',
'url', 'headline' and 'uncertainty';

\item 20-21: iterates through directories and obtains a list of files in
each folder;

\item 23-26: opens and counts each article;

\item 27-30: excludes articles containing the word ``SPORT'' among the tags;

\item 31-40: if the article body contains one of the terms in line 11, the
date, url, tags and a ``1'' binary indicator are written to a csv file;

\item 41-49: otherwise, date, url, tags and a ``0'' binary indicator are
written to the same csv file;

\item 50-53: handles csvError exceptions (i.e. ignores files in folder that
are not csv format);

\item 57-64: the tagged article list just obtained is loaded in a Pandas
dataframe, dates are converted to a machine-readable format and daily
frequencies are obtained and written to a csv file; the actual index is
computed at line 62.
\end{itemize}

\begin{lstlisting}[language=Python]
import time
import os
import csv
import re
import pandas as pd

start_time = time.time()

article_count = 0

combined_semantic = "\\risk\\b|\\brisks\\b|\\brisks\\b"

with open('art_list_RISK.csv', 'w', encoding='utf-8') as el:
    spamwriter = csv.DictWriter(el, delimiter=',', fieldnames=['date', 'url', 'alt_url', 'tags', 'filename',
                                                               'uncertainty'],
                                lineterminator='\n')
    spamwriter.writeheader()

# search iteration
for root, dirs, filename in os.walk(r'/home/mattia/Desktop/All_articles'):
    for sub_file in filename:
        try:
            with open(os.path.join(root, sub_file), 'r', encoding="utf8", errors='ignore') as f:
                spamreader = csv.DictReader(f, delimiter='|', fieldnames=['date', 'longdate', 'url', 'alt_url', 'header',
                                                                          'article', 'tags'])
                article_count += 1
                for row in spamreader:
                    if 'SPORT' in [x.strip() for x in row['tags'].split(',')]:
                        print('sport', row['url'])
                        break
                match = re.search(combined_semantic, row['article'], re.IGNORECASE)
                if match:
                    print('progress', '%.3f' % ((article_count / 2803677) * 100), '%')
                    line_with_dummy = dict(url=row['url'], alt_url=row['alt_url'], date=row['date'],
                                           tags=row['tags'], filename=sub_file, uncertainty='1')
                    with open('art_list_RISK.csv', 'a', encoding='UTF-8') as out:
                        spamwriter_2 = csv.DictWriter(out, delimiter=',', fieldnames=['date', 'url', 'alt_url', 'tags',
                                                                                      'filename', 'uncertainty'],
                                                      lineterminator='\n')
                        spamwriter_2.writerow(line_with_dummy)
                else:
                    row.update({'uncertainty': '0'})
                    line_with_zero = dict(url=row['url'], alt_url=row['alt_url'], date=row['date'],
                                          tags=row['tags'], filename=sub_file, uncertainty='0')
                    with open('art_list_RISK.csv', 'a', encoding='UTF-8') as out:
                        spamwriter_3 = csv.DictWriter(out, delimiter=',', fieldnames=['date', 'url', 'alt_url', 'tags',
                                                                                      'filename', 'uncertainty'],
                                                      lineterminator='\n')
                        spamwriter_3.writerow(line_with_zero)
        except csv.Error:
            with open('csverror.txt', 'a', encoding='utf-8') as csverr:
                csverr.write(root + sub_file)
                pass

print('end list article list')

df = pd.read_csv('art_list_RISK.csv')
print('dataset in memory')
df["date"] = pd.to_datetime(df["date"])
frequency_table = pd.crosstab(index=df["date"], columns=df['uncertainty'], margins=True)
df2 = pd.DataFrame(frequency_table)
df2['ratio'] = df2[1]/df2['All']
df2.columns = ['no_uncertainty', 'uncertainty', 'all', 'ratio']
df2.to_csv('Marcellino_daily_RISK_index.csv', encoding='utf-8')

elapsed_time = time.time() - start_time

print(elapsed_time/60, 'minutes')
\end{lstlisting}

\section{Task 3. File name: \color{blue}{Outliers.R}}

In this task we were mostly concerned with outliers detection, seasonalities
and data cleaning. Below, we present the functions we use in the \emph{%
``Outliers.R"} file.

\begin{lstlisting}[title=\textbf{Using \emph{scores()} from the \emph{``outliers"} package.}]
# Input: x is a vector of data, unstructured or
#          aggregated depending on the theme.
#        type: "z" calculates normal scores
#               (differences between each value and
#               the mean divided by sd)
#       prob: the corresponding p-values are returned
#               level choice depends on the user
od <- scores(x, type="z", prob=0.99)

# identify the position of outliers
which(od==TRUE)
\end{lstlisting}

Now, we use the built-in \emph{``stl"} to identify and extract the trend and
seasonal component. This will lead to the cleaned series.
\begin{lstlisting}[title=\textbf{Seasonal Decomposition of Time Series by Loess.}]
# Input: aggx is a numeric vector of
#        aggregated time series in weekly frequency
# 		 First, we transform the numeric vector in
#        a time series (ts) object correctly
#        specifying the frequency.
tsaggx <- ts(aggx, frequency=7)

# Then, we use the ts object, tsaggx, as
#       the input in the LOESS function.
# 	s.window: can be a string "periodic"
#       or "per" which reads the frequency
#       from the ts transformation, otherwise
#       it can be a user choice.
ss <- stl(tsaggx, s.window="per")

# Plot the output
plot(ss, main="Daily Aggregation, Weekly Pattern")

# Extract the seasonal component (xs)
#   and the trend component (xt)
xs <- ss$time.series[,1]
xt <- ss$time.series[,2]

# Calculate the cleaned series (xc)
xc <- tsaggx-xs-xt

# And create a plot
plot(xc, type="l", xlab="", ylab="", main="Detrended and Deseasonalised")
\end{lstlisting}

\section{Task 4}

In Section 4, we present the codes for Ridge, Lasso, Elastic Net penalised regressions, regression trees and random forests. For the penalised regression models, we use several functions from the \emph{``glmnet"} package.
\subsection{Penalised Regression Models. File name: \color{blue}{Ridge.R, ElasticNet.R}}

For the estimation of Ridge and Elastic Net regression, we use the \emph{``glmnet"} package. The lasso model is discussed later in Section 4.10. The main function is glmnet(x, y, family=".", alpha=.) and it implements penalised regression of an N by p matrix of explanatory variables x on a N-dimensional vector y.

The package also performs k-fold cross validation, with the function cv.glmnet. Finally, to generate predictions the following command can be used: predict(fit.info, newdata=x.test), where fit.info is the output from the glmnet (e.g. coefficients/ confidence intervals) etc.


\begin{lstlisting}[title=\textbf{Ridge Penalised Regression.}]
# Example Code Ridge
rm(list=ls())
install.packages("glmnet") # download and install package

library (glmnet)


#generate some artificial data

set.seed(1)

n <- 200  # Number of observations

p <- 300  # Number of predictors included in model

beta<- c(1/(1:p)^2)
 #approximately sparse model, slope coefficients small but not zero
x <- matrix(rnorm(n*p), nrow=n, ncol=p)

y <- x%*%beta + rnorm(n)
 #generate the dependant variable y

fit.ridge <- glmnet(x, y, family="gaussian", alpha=0)
#select ridge penalty

nforecast=5
xnew <- matrix(rnorm(nforecast*p), nrow= nforecast, ncol=p)

predict(fit.ridge, newdata=xnew) # generate predictions based on estimated coefficients

\end{lstlisting}

\begin{lstlisting}[title=\textbf{Lasso Regression.}]

# Example Code Lasso
rm(list=ls())
install.packages("glmnet") # download and install package

library (glmnet)


#generate some artificial data

set.seed(1)

n <- 200  # Number of observations

p <- 300  # Number of predictors included in model

beta<- matrix(c(rep(1,p/2),rep(0,p/2)))
#sparse model, some slope coefficients are zero
x <- matrix(rnorm(n*p), nrow=n, ncol=p)

y <- x%*%beta + rnorm(n)
# generate the dependant variable y

fit.lasso <- glmnet(x, y, family="gaussian", alpha=1) #select Lasso penalty


nforecast=5
xnew <- matrix(rnorm(nforecast*p), nrow= nforecast, ncol=p)

predict(fit.lasso, newdata=xnew) # generate predictions based on estimated coefficients

\end{lstlisting}

\begin{lstlisting}[title=\textbf{Elastic Net Regression.}]

# Example Code Elastic Net
rm(list=ls())
install.packages("glmnet") # download and install package

library (glmnet)


#generate some artificial data

set.seed(1)

n <- 200  # Number of observations

p <- 300  # Number of predictors included in model

beta1<- c(1/(1:p/2)^2)
beta<- matrix(c(beta1,rep(0,p/2)))
#combianation between sparse and approximately sparse coefficient vector
x <- matrix(rnorm(n*p), nrow=n, ncol=p)

y <- x%*%beta’ + rnorm(n)
# generate the dependant variable y

fit.elnet <- glmnet(x, y, family="gaussian", alpha=0.4) #gives 40% weight to Lasso penalty

nforecast=5
xnew <- matrix(rnorm(nforecast*p), nrow= nforecast, ncol=p)

predict(fit.elnet, newdata=xnew) # generate predictions based on estimated coefficients

\end{lstlisting}

\subsection{Spike and Slab Regression. File name: \color{blue}{SpikeSlab.R}}

For the Spike and Slab regression model, we use the \emph{``BoomSpikeSlab"} package. The main function is \emph{``lm.spike(y~x, iter)" }where x is an N by p matrix of explanatory variables, y is an N-dimensional vector and iter is the number of Metropolis draws from the posterior distribution of the parameters.

\begin{lstlisting}[title=\textbf{Slab and Spike regression.}]

# Example Code Slab and Spike
rm(list=ls())
install.packages("BoomSpikeSlab") # download and install package

library (BoomSpikeSlab)

#generate some artificial data

set.seed(1)

n = 200 #sample size

p = 300 # number of variables

nonzerob = 3 # nubmer of variables with non-zero coefficients

niter <- 1000 # nubmer of MCMC draws

sigma <- .8

x <- cbind(1, matrix(rnorm(n * (p-1)), nrow=n))

beta <- c(rep(2,ngood),rep(0, p-nonzerob ))


y <- rnorm(n, x %*% beta, sigma)

x <- x[,-1]


#Estimate spike and Slab regression
model <- lm.spike(y ~ x, niter=niter)


#Plots of coefficients
plot.ts(model$beta)

hist(model$sigma) ## should be near 8

plot(model)

summary(model)


#Plot residuals
plot(model, "residuals")


Xnew = cbind( matrix(rnorm(n * (p-1)), nrow=n))


#if out-of-sample forecasts are required
yhat.slab.new = predict.lm.spike(model, newdata=Xnew) #out-of-sample prediction


\end{lstlisting}


\subsection{Regression Trees and Forests. File name: \color{blue}{Tree.R, Forest.R}}

For the regression trees and forests, we make use of four R packages \emph{``ISLS"}, \emph{``randomForest"}, \emph{``rpart"} and \emph{``rpart.plot"}. To implement a standard regression tree with default settings:
\\
\emph{``fit.trees<- rpart(y~x)"}
\\
where x is an N by p matrix of explanatory variables and y is N-dimensional vector.

To prune a tree, the following command can be used:

\begin{lstlisting}[title=\textbf{Tree Pruning}]
bestcp        <- trees$cptable[which.min(trees$cptable[,"xerror"]),"CP"]
fit.prunedtree          <- prune(fit.trees,cp=bestcp)
prp(fit.prunedtree)
\end{lstlisting}

The main function to optimally estimate a forest is \emph{``RFfit<- tuneRF(x, y)"}.
Finally, the packages can be used to generate predictions. For regression trees, this can be achieved with the command:
\begin{lstlisting}[title=\textbf{Main Command}]
yhat.pt<- predict(fit.prunedtree, newdata=as.data.frame(..)).
\end{lstlisting}

For forests, this can be achieved with the command:
\begin{lstlisting}[title=\textbf{Main Command}]
yhat.rf2<- predict(RFfit, newdata=(..)).
\end{lstlisting}

Below, we illustrate with an example.

\begin{lstlisting}[title=\textbf{Standard Regression Tree}]
rm(list=ls())

# Regression Tree

# download and install packages
install.packages("ISLR") # download and install package
install.packages("randomForest") # download and install package
install.packages("rpart") # download and install package
install.packages("rpart.plot") # download and install package

library (ISLR)

library(randomForest)

library(rpart)

library(rpart.plot)


#generate some artificial data

set.seed(1)


n <- 200  # Number of observations

p <- 300  # Number of predictors included in model

beta<- c(10/(1:p)^2)

x <- matrix(rnorm(n*p), nrow=n, ncol=p)

y <- x%*%beta + rnorm(n)*4


#Estimate Standard Regression tree
on the atrificial data
fit.trees<- rpart(y~x)

prp(fit.trees)


#Estimate pruned Regression tree on the atrificial data
bestcp        <- trees$cptable[which.min(trees$cptable[,"xerror"]),"CP"]

fit.prunedtree          <- prune(fit.trees,cp=bestcp)

prp(fit.prunedtree)


\end{lstlisting}

\begin{lstlisting}[title=\textbf{Random Forest}]
rm(list=ls())

# Random Forest

# download and install packages
install.packages("ISLR") # download and install package
install.packages("randomForest") # download and install package
install.packages("rpart") # download and install package
install.packages("rpart.plot") # download and install package

library (ISLR)

library(randomForest)

library(rpart)

library(rpart.plot)


#generate some artificial data

set.seed(1)


n <- 200  # Number of observations

p <- 300  # Number of predictors included in model

beta<- c(10/(1:p)^2)

x <- matrix(rnorm(n*p), nrow=n, ncol=p)

y <- x%*%beta + rnorm(n)*4


#Estimate a Random forest on the atrificial data

RFfit<- tuneRF(x, y, mtryStart=floor(sqrt(ncol(x))),stepFactor=1.5, improve=0.05, nodesize=5, ntree=2000, doBest=TRUE)


#Find the best fir for the Random forest on the atrificial data

min             <- RFfit$mtry

fit.rf2 <-randomForest(x, y, nodesize=5, mtry=min, ntree=2000)


\end{lstlisting}

\subsection{Bayesian VAR models. File name: \color{blue}{BVAR.R}}

We make use the of the \emph{``MSBVAR"} package for Bayesian vectorautoregressive models. The main function is szbvar, which takes the following inputs: Y is T by M matrix of time series; p is lag length; z T by N matrix of exogenous vars, can be NULL; lambda0 between 0 and 1, overall tightness; lambda1 is the standard deviation or tightness of the prior around the AR(1) parameters; lambda3 Lag decay (> 0, with 1=harmonic); lambda4 Standard deviation or tightness around the intercept > 0; lambda5 is the standard deviation or tightness around the exogneous variable coefficients; mu5 is the sum of coefficients prior weight (larger values imply difference stationarity); mu6 is dummy initial observations or drift prior (larger values allow for common trends); nu is the prior degrees of freedom, m + 1; qm is the frequency of the data for lag decay equivalence; the input prior can be of three values: 0 = Normal-Wishart prior, 1 = Normal-flat prior, 2 = flat-flat prior; posterior.fit is a logical, FALSE implies no estimation of log-posterior fit measures.
The package can be used to generate out-of-sample forecasts, using the function forecast, for example:
forecasts <- forecast(fit.bvar, nsteps)) with nsteps the numbers of horizons for the out-of-sample forecasts.

\begin{lstlisting}[title=\textbf{Bayesian VAR}]
rm(list=ls())

# Bayesian VAR
install.packages("MSBVAR") # download and install package

library (MSBVAR)


#generate some artificial data

set.seed(1)

n = 200 #sample size

p = 10 # number of variables

X0 = rep(0,p)

beta = rep(0.5,p)

B=diag(beta)

y=matrix(0,nrow=p,ncol=n)

for (i in 1:n) {

e = rnorm(p)

y[,i]=B%*%X0+e

X0=y[,i]

}


# Reference prior model -- Normal-IW prior pdf

Bvar.Model <- szbvar(y, p=6, z=NULL, lambda0=0.6,

lambda1=0.1, lambda3=2, lambda4=0.5,

lambda5=0, mu5=0, mu6=0,

nu=ncol(KEDS)+1, qm=4, prior=0,

posterior.fit=F)


# Forecast -- this gives back the sample PLUS the forecasts.
forecasts <- forecast(Bvar.Model, nsteps=10,burnin=3000, gibbs=5000, exog=NULL)


# Conditional forecasts

conditional.forcs.ref <- hc.forecast(Bvar.Model, yhat, nsteps,

burnin=3000, gibbs=5000, exog=NULL)

\end{lstlisting}

\section{Tasks 5 $-$ 8}

In Tasks 5, 6 and 7 we divide the general codes in three parts: (i) data
manipulation, (ii) nowcasting and (iii) post-processing of results.

The first part refers to all codes we used to manipulate the downloaded
data. This part is ``data-specific" and it applies to data downloaded from
the same sources as in these tasks. If a researchers uses Bloomberg,
Reuters, Macrobond or other data collection software, the original data is
different and cannot be used as input in these functions. Therefore, we do
not discuss them here. The following sections, we take it as granted that the user
has already downloaded and cleaned the data.

\subsection{Weekly Google Trends. File name: \color{blue}{Weekly-Google.R}}

As mentioned in a previous section, Google Trends are downloaded in monthly
frequency. We have developed a function which loads Google Trends over
smaller time frames at weekly frequency, and then scales the data in order
to obtain the weekly Google Trends for the overall period. The function uses
the main function from \emph{``gtrendsR"} package.

\begin{lstlisting}[title=\textbf{Weekly Google Trends.}]
# Set dates for all data
dfrom <- "2004-01-01"	# starting date
dto <- "2017-09-01"		# ending date

#  (the news doesn't really have a lot of data, so stick to the web)
stp <- "web"            # web; news;
cct <- 0                # category

# General Indexes - web
reg <- ""               # Region, blank for all regions or use "GB" for the UK, etc.
kwd <- "uncertainty"

# Download the weekly trends and store them in c1 variable
# Input: kwd (keyword), reg (region)
#		 cct (category), stp (domain)
#		 dfrom (start), dto (end)
c1 <- weekly.GOOGLE(kwd, reg, cct, stp, dfrom, dto)
\end{lstlisting}

\subsection{Transformation}

In ``averaging.R", ..., we transform the final nowcast/forecast estimates to
levels according to the nature of the series. In order to avoid repetition,
we describe the code here.

\begin{lstlisting}[title=\textbf{From Change to Levels.}]
# YTRANSF: if 3, we translate from growth to levels
#		   if 2, we translate from 1-st diff to levels
#
#	zlast: is the last observed value for the dependent variable
#	zboot: are the bootstrap estimate for densities
# 	zave: (or different name) is the final estimate in levels.
if(YTRANSF==3){
  zlast <- YSAV[NROW(YSAV)]
  zboot <- zlast*(1+zboot)
  zave <- zlast*(1+zave)
}
if(YTRANSF==2){
  zlast <- YSAV[NROW(YSAV)]
  zboot <- zlast +zboot
  zave <- zlast + zave
}
\end{lstlisting}

\subsection{Averaging. File name: \color{blue}{averaging.R}}

We calculate the average value of the last observations and also use bootstrap
to calculate the corresponding density estimates.

\begin{lstlisting}[title=\textbf{Averaging and Bootstrap for Density.}]
# Input: "z" is the dependent variable, vector of data
#			zp: window length, e.g. zp=4, then we have the 4-period MA.
zave <- mean(z[(NROW(z)-zp+1):NROW(z),])

# Bootstrap for density estimation
# b: bootstrap window length
# B: number of bootstraps
b <- round(NROW(z)^(1/3))
boots <- matrix(NA, NROW(z), B)
for(j in 1:B){
  boots[,j] <- MBB(z, b) # MBB: Moving Block Bootstrap
}
# Calculate the mean value for the same zp
zboot <- colMeans(boots[(NROW(boots)-zp+1):NROW(boots),])
\end{lstlisting}

\subsection{Naive Estimate. File name: \color{blue}{naive.R}}

We calculate the naive forecast and also use bootstrap to calculate the
corresponding density estimates.

\begin{lstlisting}[title=\textbf{Naive Forecasting.}]
# Input: "z" is the dependent variable, vector of data
# 		here we extract the last observed value
zf <- z[NROW(z)]

# Bootstrap for density estimation
# b: bootstrap window length
# B: number of bootstraps
b <- round(NROW(z)^(1/3))
boots <- matrix(NA, NROW(z), B)
for(j in 1:B){
  boots[,j] <- MBB(z, b)
}
# Extract the corresponding naives
zboot <- boots[NROW(boots),]
\end{lstlisting}

\subsection{ARIMA. File name: \color{blue}{ar.R}}

We calculate various ARIMA model-based forecasts using the \emph{``forecast"}
package.

\begin{lstlisting}[title=\textbf{ARIMA Forecasting.}]
# If an AR order is set to zero, we use AIC
if(arp==0){ arp <- NROW(ar.ols(z, aic=TRUE)$ar) }

# Estimate ARIMA using
#	z: the vector of the observed dependent variable
# order: ARIMA order
# method: conditional sum of squares
fit <- Arima(z,order=c(arp,0,0), method=c("CSS"))
fout <- NULL

# Calculate percentiles
try(fout <- forecast(fit,h=1, bootstrap=TRUE, npaths=B, level=seq(51,99,1)), silent=TRUE)

# Put all percentiles together in the correct order
zout <- c(fout$mean, rev(as.numeric(fout$lower))[1], rev(fout$lower),
          fout$mean, fout$upper, as.numeric(fout$upper)[49])
\end{lstlisting}

\subsection{Dynamic Factor Analysis. File name: \color{blue}{dfa.R}}

We calculate forecasts/nowcasts based on Dynamic Factor Analysis. Codes are
adapted versions of Giannone and Reichlin original MATLAB programs.

\begin{lstlisting}[title=\textbf{Dynamic Factor Analysis Forecasting.}]
# Extract factors using:
#	XX: matrix of observed regressors
#	q: dynamic rank
#   r: static rank (r>=q)
#   p: ar order of the state vector
fac.out <- FactorExtraction(XX,q=fq,r=fr,p=fp)
Fac <- fac.out$Fac
rownames(Fac) <- rownames(XX)

# Then use the linreg with the factors
z <- YY; f <- Fac; vlag <- 0; source(".../linreg.R")
\end{lstlisting}

\subsection{Factor Linear Regression. File name: \color{blue}{Flinreg.R}}

We calculate forecasts/nowcasts based on linear regression using factors which
come from standardised matrices.

\begin{lstlisting}[title=\textbf{Factor Linear Regression.}]
# jj: step-ahead
# z: target, vector
# f: factor which comes from standardised data
# ysd: the standard deviation of target
# ymu: the mean of target
jj <- 1
zreg <- as.matrix(z[(jj+1):NROW(z),])
freg <- as.matrix(f[1:(NROW(f)-jj),])
out <- lm(zreg~freg)
b <- out$coefficients;

# Now make sure to use ysd, ymu as we used factors from standardised input
outf <- ((f[NROW(f),]%*%b[2:NROW(b)]) + b[1])*ysd+ymu
\end{lstlisting}

\subsection{Partial Least Squares. File name: \color{blue}{pls.R}}

We calculate forecasts/nowcasts based on partial least squares methodology. We
are using the \emph{``plsr"} package.

\begin{lstlisting}[title=\textbf{Factor Linear Regression.}]
# Lag the matrix of Regressors if necessary
# vlag: lag order
# XX : matrix, panel of regressors
# YY : vector, observed target
vlag <- 1
if(vlag>0){
  XX <- cbind(lagf(YY, vlag)[,2:(vlag+1)], XX)

  XX <- as.matrix(XX[(vlag+1):NROW(XX),])
  YY <- as.matrix(YY[(vlag+1):NROW(YY),])
}

# Standardise
xxin <- xstd(XX)
ymu <- mean(YY)
ysd <- sd(YY)
yyin <- (YY-ymu)/ysd

# Extract factors
pp <- plsr(yyin~xxin, ncomp=qncomp, scale=FALSE)
f <- as.matrix(pp$scores)
z <- as.matrix(yyin)

# Use Factor linear regression
source(".../Flinreg.R")
\end{lstlisting}

\subsection{Sparse Principal Components. File name: \color{blue}{spc.R}}

We calculate forecasts/nowcasts based on sparse principal components
methodology. We are using the \emph{``nsprcomp"} package.

\begin{lstlisting}[title=\textbf{Sparse Principal Components.}]
# jj: step-ahead
# z: target, vector
# f: factor which comes from standardised data
# ysd: the standard deviation of target
# ymu: the mean of target
vlag <- 1
if(vlag>0){
  # XX <- lagmv(XX, vlag)   # no because of small T dimension
  XX <- cbind(lagf(YY, vlag)[,2:(vlag+1)], XX)

  XX <- as.matrix(XX[(vlag+1):NROW(XX),])
  YY <- as.matrix(YY[(vlag+1):NROW(YY),])
}

# Standardise
xxin <- xstd(XX)
ymu <- mean(YY)
ysd <- sd(YY)
yyin <- (YY-ymu)/ysd

# Extract factors
pc.out <- nsprcomp(x=xxin, retx=TRUE, ncomp=qncomp, nneg = FALSE, center=FALSE, scale.=FALSE)
f <- as.matrix(pc.out$x)
z <- as.matrix(yyin)

# Use Factor linear regression
source(".../Flinreg.R")
\end{lstlisting}

\subsection{Sparse Regression. File name: \color{blue}{sparse.R}}

We calculate forecasts/nowcasts based on sparse regression methodology. We are
using the \emph{``glmnet"} package.

\begin{lstlisting}[title=\textbf{Sparse Regression.}]
# jj: step-ahead
# z: target, vector
# f: factor which comes from standardised data
# ysd: the standard deviation of target
# ymu: the mean of target
vlag <- 1
if(vlag>0){
  XX <- cbind(lagf(YY, vlag)[,2:(vlag+1)], XX)

  XX <- as.matrix(XX[(vlag+1):NROW(XX),])
  YY <- as.matrix(YY[(vlag+1):NROW(YY),])
}
z <- YY
f <- XX

# jj: step ahead, then correctly lead/lag the variables
jj <- 1
zreg <- as.matrix(z[(jj+1):NROW(z),])
freg <- as.matrix(f[1:(NROW(f)-jj),])

# Calculate beta which comes from sparse
# freg: regressors,  matrix
# zreg: target, vector
# type.measure: "mse" for the calculation of lambda
# alpha: 1 for Lasso, 0.5 for LAR
fit.lasso.cv <- cv.glmnet(freg, zreg, type.measure="mse", alpha=salpha, family="gaussian", standardize=TRUE)
s <- fit.lasso.cv$lambda.min
b <- as.numeric(coef(fit.lasso.cv, s))
outf <- ((f[NROW(f),]%*%b[2:NROW(b)]) + b[1])

# Calculate percentiles
sigmah <- sd(zreg-freg%*%b[2:NROW(b)]-b[1])
spseq <- qnorm(seq(0.51, 0.99, 0.01))*sigmah
zout <- c(outf, outf-rev(spseq)[1], outf-rev(spseq), outf, outf+spseq, outf+spseq[NROW(spseq)])
\end{lstlisting}

\subsection{Spike and Slab. File name: \color{blue}{sparse.R}}

We calculate forecasts/nowcasts based on sparse regression methodology. We are
using the \emph{``BoomSpikeSlab"} package.

\begin{lstlisting}[title=\textbf{Spike and Slab Regression.}]
# jj: step-ahead
# z: target, vector
# f: factor which comes from standardised data
# ysd: the standard deviation of target
# ymu: the mean of target
lag <- 1
if(vlag>0){
  # XX <- lagmv(XX, vlag)   # no because of small T dimension
  XX <- cbind(lagf(YY, vlag)[,2:(vlag+1)], XX)

  XX <- as.matrix(XX[(vlag+1):NROW(XX),])
  YY <- as.matrix(YY[(vlag+1):NROW(YY),])
}
# Standardise
xxin <- xstd(XX)
ymu <- mean(YY)
ysd <- sd(YY)
yyin <- (YY-ymu)/ysd

z <- yyin
f <-xxin

jj <- 1
zreg <- as.matrix(z[(jj+1):NROW(z),])
freg <- as.matrix(f[1:(NROW(f)-jj),])
# Spike and Slab
# niter: The number of MCMC iterations to run
# ping: output printing parameter
out <- lm.spike(zreg ~ freg, niter=niters, ping=B)
# keep the last B rounds
b <- out$beta
b <- b[(NROW(b)-B+1):NROW(b),]
bf <- colMeans(b)
outf <- ((f[NROW(f),]%*%bf[2:NROW(bf)]) + bf[1])*ysd+ymu
#Calculate percentiles
sigmah <- sd((zreg-freg%*%bf[2:NROW(bf)]-bf[1])*ysd+ymu)
spseq <- qnorm(seq(0.51, 0.99, 0.01))*sigmah
zout <- c(outf, outf-rev(spseq)[1], outf-rev(spseq), outf, outf+spseq, outf+spseq[NROW(spseq)])
\end{lstlisting}

\subsection{Evaluation Statistics}

We calculate various evaluation statistics. Check the main files for more
details.

\begin{lstlisting}[title=\textbf{Mean Absolute Error.}]
# err: matrix with errors of various methods
mae <- as.matrix(colMeans(abs(err)))

# relative MAE
benchmark <- "AR(1)"
maeR <- mae/mae[benchmark,1]
\end{lstlisting}

\begin{lstlisting}[title=\textbf{Root Mean Squared Forecast Error.}]
# err: matrix with errors of various methods
rmsfe <- as.matrix(sqrt(colMeans(err^2)))

# relative RMSFE
benchmark <- "AR(1)"
rmsfeR <- rmsfe/rmsfe[benchmark,1]
\end{lstlisting}

\begin{lstlisting}[title=\textbf{Two Sided Diebold-Mariano.}]
# Using the package ``forecast"
# err: matrix with errors of various methods
# i:  method i stored in column i of err
# h: horizon
# power: power
benchmark <- "AR(1)"
dmp <- dm.test(err[,i], err[,benchmark], alternative=c("two.sided"), h=1, power=2)

# Extract the p-value
dmp <- as.numeric(dmp$p.value)
\end{lstlisting}

\begin{lstlisting}[title=\textbf{Sign Success Ratio.}]
# err: matrix with errors of various methods
# forc: signs of forecast direction compared to the last period for a given method
# sgnt: signs of direction of the target
ssr <- sum(sgnt==sign(forc))

# relative SSR
benchmark <- "AR(1)"
ssrR <- ssr/ssr[benchmark,1]
\end{lstlisting}

\begin{lstlisting}[title=\textbf{Berkowitz LR Test.}]
# xcloud: all percentile forecasts
xcloud <- allf
z <- pnorm(xcloud[,1], mean=apply(xcloud[,2:NCOL(xcloud)], 1, mean), sd=apply(xcloud[,2:NCOL(xcloud)], 1, sd))
Z <- qnorm(z)

# Extract Berkowitz P-value
berk <- BerkowitzTest(Z, lags=1, significance = 0.05)$LRp
\end{lstlisting}

\end{document}
